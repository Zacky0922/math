\documentclass[b4paper,twocolumn]{jsarticle}
\pagestyle{empty}
\def\labelenumi{\textbf{\theenumi.}}
\def\labelenumii{(\theenumii)}
\def\theenumii{\arabic{enumii}}
\setlength{\columnseprule}{0.4truept}\setlength{\columnsep}{1zw}
\setlength{\topmargin}{-1in}
\setlength{\oddsidemargin}{-0.75in}
\makeatletter
\def\le{\mathrel{\mathpalette\gl@align<}}
\def\ge{\mathrel{\mathpalette\gl@align>}}
\def\gl@align#1#2{\lower.6ex\vbox{\baselineskip\z@skip\lineskip\z@
\ialign{$\m@th#1\hfil##\hfil$\crcr#2\crcr=\crcr}}}
\def\nfrac#1#2{%
\displaystyle \frac{\lower.44ex\hbox{$\,#1\,$}}{\lower-.1ex\hbox{$\,#2\,$}}}%
\def\zahyo#1#2{\left( \,#1,~#2\, \right)}



\begin{document}

\twocolumn[
{\Large \bf 三角比のまとめ1}
\hfill
    年  組  番 氏名\underline{\hspace{50truemm}}
\vspace{1zh}]

\begin{enumerate}
\item 次の三角比の値を求めよ。

\begin{enumerate}
\item $\tan 0^\circ $
\vfill

\hfill 答.\underline{\hspace{50truemm}}

\item $\cos 180^\circ $
\vfill

\hfill 答.\underline{\hspace{50truemm}}

\item $\tan 120^\circ $
\vfill

\hfill 答.\underline{\hspace{50truemm}}

\item $\sin 150^\circ $
\vfill

\hfill 答.\underline{\hspace{50truemm}}

\item $\cos 60^\circ $
\vfill

\hfill 答.\underline{\hspace{50truemm}}

\item $\sin 120^\circ $
\vfill

\hfill 答.\underline{\hspace{50truemm}}

\end{enumerate}

\item 次の三角比を、(  )内の大きさの角の三角比で表せ。

\begin{enumerate}
\item $\cos 53^\circ$ \hfill ($45^\circ$以下) \hspace{30truemm}
\vfill

\hfill 答.\underline{\hspace{50truemm}}

\item $\sin 74^\circ$ \hfill ($45^\circ$以下) \hspace{30truemm}
\vfill

\hfill 答.\underline{\hspace{50truemm}}

\end{enumerate}

\newpage

\item $\triangle$ABCについて、次の問いに答えよ。

\begin{enumerate}
\item $\sin \theta =\nfrac{\sqrt{\,6\,}}{4}のとき、\cos \theta 、\tan \theta の値を求めよ。(0^\circ \le \theta \le 90^\circ )$
\vfill

\hfill 答.$\cos \theta$=\underline{\hspace{25truemm}}\,,\,$\tan \theta$=\underline{\hspace{25truemm}}

\item $\sin \theta =\nfrac{1}{4}のとき、\cos \theta 、\tan \theta の値を求めよ。(0^\circ \le \theta \le 180^\circ )$
\vfill

\hfill 答.$\cos \theta$=\underline{\hspace{25truemm}}\,,\,$\tan \theta$=\underline{\hspace{25truemm}}

\end{enumerate}

\item 次のような三角形ABCで、(  )内の値を求めよ。

\begin{enumerate}
\item $B=30^\circ  \,,\, R=5 \hfill (\,b\,) \hspace{30truemm}$
\vfill

\hfill 答.\underline{\hspace{50truemm}}

\end{enumerate}

\item 次のような三角形ABCで、(  )内の値を求めよ。

\begin{enumerate}
\item $a=2\sqrt{\,2\,} \,,\, b=5 \,,\, C=135^\circ  \hfill (\,c\,) \hspace{30truemm}$
\vfill

\hfill 答.\underline{\hspace{50truemm}}

\end{enumerate}

\item 次のような三角形ABCの面積を求めよ。

\begin{enumerate}
\item $a=2 \,,\, c=4 \,,\, B=120^\circ $
\vfill

\hfill 答.\underline{\hspace{50truemm}}

\end{enumerate}

\end{enumerate}

\newpage

\twocolumn[
{\Large \bf 三角比のまとめ1の解答}
\hfill
    年  組  番 氏名\underline{\hspace{50truemm}}
\vspace{1zh}]

\begin{enumerate}
\item 次の三角比の値を求めよ。

\begin{enumerate}
\item $\tan 0^\circ $
\vfill

\hfill 答.$0$

\item $\cos 180^\circ $
\vfill

\hfill 答.$-1$

\item $\tan 120^\circ $
\vfill

\hfill 答.$-\sqrt{\,3\,}$

\item $\sin 150^\circ $
\vfill

\hfill 答.$\nfrac{1}{2}$

\item $\cos 60^\circ $
\vfill

\hfill 答.$\nfrac{1}{2}$

\item $\sin 120^\circ $
\vfill

\hfill 答.$\nfrac{\sqrt{\,3\,}}{2}$

\end{enumerate}

\item 次の三角比を、(  )内の大きさの角の三角比で表せ。

\begin{enumerate}
\item $\cos 53^\circ$ \hfill ($45^\circ$以下) \hspace{30truemm}
\vfill

\hfill 答.$\sin 37^\circ$

\item $\sin 74^\circ$ \hfill ($45^\circ$以下) \hspace{30truemm}
\vfill

\hfill 答.$\cos 16^\circ$

\end{enumerate}

\newpage
\item $\triangle$ABCについて、次の問いに答えよ。

\begin{enumerate}
\item $\sin \theta =\nfrac{\sqrt{\,6\,}}{4}のとき、\cos \theta 、\tan \theta の値を求めよ。(0^\circ \le \theta \le 90^\circ )$
\vfill

\begin{flushright}
答.$\cos \theta =\nfrac{\sqrt{\,10\,}}{4}\,,\,\tan \theta =\nfrac{\sqrt{\,3\,}}{\sqrt{\,5\,}}$
\end{flushright}

\item $\sin \theta =\nfrac{1}{4}のとき、\cos \theta 、\tan \theta の値を求めよ。(0^\circ \le \theta \le 180^\circ )$
\vfill

\begin{flushright}
答.$\cos \theta =\pm \nfrac{\sqrt{\,15\,}}{4}\,,\,\tan \theta =\pm \nfrac{1}{\sqrt{\,15\,}}$
\end{flushright}


\end{enumerate}

\item 次のような三角形ABCで、(  )内の値を求めよ。

\begin{enumerate}
\item $B=30^\circ  \,,\, R=5 \hfill (\,b\,) \hspace{30truemm}$
\vfill

\hfill 答.$5$

\end{enumerate}

\item 次のような三角形ABCで、(  )内の値を求めよ。

\begin{enumerate}
\item $a=2\sqrt{\,2\,} \,,\, b=5 \,,\, C=135^\circ  \hfill (\,c\,) \hspace{30truemm}$
\vfill

\hfill 答.$\sqrt{\,53\,}$

\end{enumerate}

\item 次のような三角形ABCの面積を求めよ。

\begin{enumerate}
\item $a=2 \,,\, c=4 \,,\, B=120^\circ $
\vfill

\hfill 答.$2\sqrt{\,3\,}$

\end{enumerate}

\end{enumerate}

\newpage

\twocolumn[
{\Large \bf 三角比のまとめ2}
\hfill
    年  組  番 氏名\underline{\hspace{50truemm}}
\vspace{1zh}]

\begin{enumerate}
\item 次の三角比の値を求めよ。

\begin{enumerate}
\item $\tan 45^\circ $
\vfill

\hfill 答.\underline{\hspace{50truemm}}

\item $\tan 30^\circ $
\vfill

\hfill 答.\underline{\hspace{50truemm}}

\item $\sin 0^\circ $
\vfill

\hfill 答.\underline{\hspace{50truemm}}

\item $\sin 45^\circ $
\vfill

\hfill 答.\underline{\hspace{50truemm}}

\item $\sin 180^\circ $
\vfill

\hfill 答.\underline{\hspace{50truemm}}

\item $\tan 150^\circ $
\vfill

\hfill 答.\underline{\hspace{50truemm}}

\end{enumerate}

\item 次の三角比を、(  )内の大きさの角の三角比で表せ。

\begin{enumerate}
\item $\cos 81^\circ$ \hfill ($45^\circ$以下) \hspace{30truemm}
\vfill

\hfill 答.\underline{\hspace{50truemm}}

\item $\sin 88^\circ$ \hfill ($45^\circ$以下) \hspace{30truemm}
\vfill

\hfill 答.\underline{\hspace{50truemm}}

\end{enumerate}

\newpage
\item $\triangle$ABCについて、次の問いに答えよ。

\begin{enumerate}
\item $\cos \theta =\nfrac{\sqrt{\,10\,}}{4}のとき、\sin \theta 、\tan \theta の値を求めよ。(0^\circ \le \theta \le 90^\circ )$
\vfill

\hfill 答.$\sin \theta$=\underline{\hspace{25truemm}}\,,\,$\tan \theta$=\underline{\hspace{25truemm}}

\item $\tan \theta =\nfrac{2\sqrt{\,2\,}}{3}のとき、\sin \theta 、\cos \theta の値を求めよ。(0^\circ \le \theta \le 180^\circ )$
\vfill

\hfill 答.$\sin \theta$=\underline{\hspace{25truemm}}\,,\,$\cos \theta$=\underline{\hspace{25truemm}}


\end{enumerate}

\item 次のような三角形ABCで、(  )内の値を求めよ。

\begin{enumerate}
\item $B=45^\circ  \,,\, R=5 \hfill (\,b\,) \hspace{30truemm}$
\vfill

\hfill 答.\underline{\hspace{50truemm}}

\end{enumerate}

\item 次のような三角形ABCで、(  )内の値を求めよ。

\begin{enumerate}
\item $a=5\sqrt{\,3\,} \,,\, c=5 \,,\, B=30^\circ  \hfill (\,b\,) \hspace{30truemm}$
\vfill

\hfill 答.\underline{\hspace{50truemm}}

\end{enumerate}

\item 次のような三角形ABCの面積を求めよ。

\begin{enumerate}
\item $a=6 \,,\, b=6 \,,\, C=135^\circ $
\vfill

\hfill 答.\underline{\hspace{50truemm}}

\end{enumerate}

\end{enumerate}

\newpage

\twocolumn[
{\Large \bf 三角比のまとめ2の解答}
\hfill
    年  組  番 氏名\underline{\hspace{50truemm}}
\vspace{1zh}]

\begin{enumerate}
\item 次の三角比の値を求めよ。

\begin{enumerate}
\item $\tan 45^\circ $
\vfill

\hfill 答.$1$

\item $\tan 30^\circ $
\vfill

\hfill 答.$\nfrac{\sqrt{\,3\,}}{3}$

\item $\sin 0^\circ $
\vfill

\hfill 答.$0$

\item $\sin 45^\circ $
\vfill

\hfill 答.$\nfrac{\sqrt{\,2\,}}{2}$

\item $\sin 180^\circ $
\vfill

\hfill 答.$0$

\item $\tan 150^\circ $
\vfill

\hfill 答.$-\nfrac{\sqrt{\,3\,}}{3}$

\end{enumerate}

\item 次の三角比を、(  )内の大きさの角の三角比で表せ。

\begin{enumerate}
\item $\cos 81^\circ$ \hfill ($45^\circ$以下) \hspace{30truemm}
\vfill

\hfill 答.$\sin 9^\circ$

\item $\sin 88^\circ$ \hfill ($45^\circ$以下) \hspace{30truemm}
\vfill

\hfill 答.$\cos 2^\circ$

\end{enumerate}

\newpage
\item $\triangle$ABCについて、次の問いに答えよ。

\begin{enumerate}
\item $\cos \theta =\nfrac{\sqrt{\,10\,}}{4}のとき、\sin \theta 、\tan \theta の値を求めよ。(0^\circ \le \theta \le 90^\circ )$
\vfill

\begin{flushright}
答.$\sin \theta =\nfrac{\sqrt{\,6\,}}{4}\,,\,\tan \theta =\nfrac{\sqrt{\,3\,}}{\sqrt{\,5\,}}$
\end{flushright}

\item $\tan \theta =\nfrac{2\sqrt{\,2\,}}{3}のとき、\sin \theta 、\cos \theta の値を求めよ。(0^\circ \le \theta \le 180^\circ )$
\vfill

\begin{flushright}
答.$\sin \theta =\nfrac{2\sqrt{\,2\,}}{\sqrt{\,17\,}}\,,\,\cos \theta =\nfrac{3}{\sqrt{\,17\,}}$
\end{flushright}


\end{enumerate}

\item 次のような三角形ABCで、(  )内の値を求めよ。

\begin{enumerate}
\item $B=45^\circ  \,,\, R=5 \hfill (\,b\,) \hspace{30truemm}$
\vfill

\hfill 答.$5\sqrt{\,2\,}$

\end{enumerate}

\item 次のような三角形ABCで、(  )内の値を求めよ。

\begin{enumerate}
\item $a=5\sqrt{\,3\,} \,,\, c=5 \,,\, B=30^\circ  \hfill (\,b\,) \hspace{30truemm}$
\vfill

\hfill 答.$5$

\end{enumerate}

\item 次のような三角形ABCの面積を求めよ。

\begin{enumerate}
\item $a=6 \,,\, b=6 \,,\, C=135^\circ $
\vfill

\hfill 答.$9\sqrt{\,2\,}$

\end{enumerate}

\end{enumerate}

\newpage

\twocolumn[
{\Large \bf 三角比のまとめ3}
\hfill
    年  組  番 氏名\underline{\hspace{50truemm}}
\vspace{1zh}]

\begin{enumerate}
\item 次の三角比の値を求めよ。

\begin{enumerate}
\item $\tan 135^\circ $
\vfill

\hfill 答.\underline{\hspace{50truemm}}

\item $\tan 45^\circ $
\vfill

\hfill 答.\underline{\hspace{50truemm}}

\item $\cos 90^\circ $
\vfill

\hfill 答.\underline{\hspace{50truemm}}

\item $\cos 30^\circ $
\vfill

\hfill 答.\underline{\hspace{50truemm}}

\item $\cos 150^\circ $
\vfill

\hfill 答.\underline{\hspace{50truemm}}

\item $\tan 150^\circ $
\vfill

\hfill 答.\underline{\hspace{50truemm}}

\end{enumerate}

\item 次の三角比を、(  )内の大きさの角の三角比で表せ。

\begin{enumerate}
\item $\sin 56^\circ$ \hfill ($45^\circ$以下) \hspace{30truemm}
\vfill

\hfill 答.\underline{\hspace{50truemm}}

\item $\sin 69^\circ$ \hfill ($45^\circ$以下) \hspace{30truemm}
\vfill

\hfill 答.\underline{\hspace{50truemm}}

\end{enumerate}

\newpage
\item $\triangle$ABCについて、次の問いに答えよ。

\begin{enumerate}
\item $\cos \theta =\nfrac{\sqrt{\,2\,}}{2}のとき、\sin \theta 、\tan \theta の値を求めよ。(0^\circ \le \theta \le 90^\circ )$
\vfill

\hfill 答.$\sin \theta$=\underline{\hspace{25truemm}}\,,\,$\tan \theta$=\underline{\hspace{25truemm}}

\item $\cos \theta =\nfrac{\sqrt{\,3\,}}{2}のとき、\sin \theta 、\tan \theta の値を求めよ。(0^\circ \le \theta \le 180^\circ )$
\vfill

\hfill 答.$\sin \theta$=\underline{\hspace{25truemm}}\,,\,$\tan \theta$=\underline{\hspace{25truemm}}


\end{enumerate}

\item 次のような三角形ABCで、(  )内の値を求めよ。

\begin{enumerate}
\item $a=5 \,,\, c=3\sqrt{\,2\,} \,,\, C=135^\circ  \hfill (\,\sin A\,) \hspace{30truemm}$
\vfill

\hfill 答.\underline{\hspace{50truemm}}

\end{enumerate}

\item 次のような三角形ABCで、(  )内の値を求めよ。

\begin{enumerate}
\item $b=3\sqrt{\,3\,} \,,\, c=2 \,,\, A=150^\circ  \hfill (\,a\,) \hspace{30truemm}$
\vfill

\hfill 答.\underline{\hspace{50truemm}}

\end{enumerate}

\item 次のような三角形ABCの面積を求めよ。

\begin{enumerate}
\item $a=6 \,,\, b=2 \,,\, C=135^\circ $
\vfill

\hfill 答.\underline{\hspace{50truemm}}

\end{enumerate}

\end{enumerate}

\newpage

\twocolumn[
{\Large \bf 三角比のまとめ3の解答}
\hfill
    年  組  番 氏名\underline{\hspace{50truemm}}
\vspace{1zh}]

\begin{enumerate}
\item 次の三角比の値を求めよ。

\begin{enumerate}
\item $\tan 135^\circ $
\vfill

\hfill 答.$-1$

\item $\tan 45^\circ $
\vfill

\hfill 答.$1$

\item $\cos 90^\circ $
\vfill

\hfill 答.$0$

\item $\cos 30^\circ $
\vfill

\hfill 答.$\nfrac{\sqrt{\,3\,}}{2}$

\item $\cos 150^\circ $
\vfill

\hfill 答.$-\nfrac{\sqrt{\,3\,}}{2}$

\item $\tan 150^\circ $
\vfill

\hfill 答.$-\nfrac{\sqrt{\,3\,}}{3}$

\end{enumerate}

\item 次の三角比を、(  )内の大きさの角の三角比で表せ。

\begin{enumerate}
\item $\sin 56^\circ$ \hfill ($45^\circ$以下) \hspace{30truemm}
\vfill

\hfill 答.$\cos 34^\circ$

\item $\sin 69^\circ$ \hfill ($45^\circ$以下) \hspace{30truemm}
\vfill

\hfill 答.$\cos 21^\circ$

\end{enumerate}

\newpage

\item $\triangle$ABCについて、次の問いに答えよ。

\begin{enumerate}
\item $\cos \theta =\nfrac{\sqrt{\,2\,}}{2}のとき、\sin \theta 、\tan \theta の値を求めよ。(0^\circ \le \theta \le 90^\circ )$
\vfill

\begin{flushright}
答.$\sin \theta =\nfrac{\sqrt{\,2\,}}{2}\,,\,\tan \theta =1$
\end{flushright}

\item $\cos \theta =\nfrac{\sqrt{\,3\,}}{2}のとき、\sin \theta 、\tan \theta の値を求めよ。(0^\circ \le \theta \le 180^\circ )$
\vfill

\begin{flushright}
答.$\sin \theta =\nfrac{1}{2}\,,\,\tan \theta =\nfrac{1}{\sqrt{\,3\,}}$
\end{flushright}

\end{enumerate}

\item 次のような三角形ABCで、(  )内の値を求めよ。

\begin{enumerate}
\item $a=5 \,,\, c=3\sqrt{\,2\,} \,,\, C=135^\circ  \hfill (\,\sin A\,) \hspace{30truemm}$
\vfill

\hfill 答.$\nfrac{5}{6}$

\end{enumerate}

\item 次のような三角形ABCで、(  )内の値を求めよ。

\begin{enumerate}
\item $b=3\sqrt{\,3\,} \,,\, c=2 \,,\, A=150^\circ  \hfill (\,a\,) \hspace{30truemm}$
\vfill

\hfill 答.$7$

\end{enumerate}

\item 次のような三角形ABCの面積を求めよ。

\begin{enumerate}
\item $a=6 \,,\, b=2 \,,\, C=135^\circ $
\vfill

\hfill 答.$3\sqrt{\,2\,}$

\end{enumerate}

\end{enumerate}

\newpage

\twocolumn[
{\Large \bf 三角比のまとめ4}
\hfill
    年  組  番 氏名\underline{\hspace{50truemm}}
\vspace{1zh}]

\begin{enumerate}
\item 次の三角比の値を求めよ。

\begin{enumerate}
\item $\cos 90^\circ $
\vfill

\hfill 答.\underline{\hspace{50truemm}}

\item $\sin 120^\circ $
\vfill

\hfill 答.\underline{\hspace{50truemm}}

\item $\cos 0^\circ $
\vfill

\hfill 答.\underline{\hspace{50truemm}}

\item $\tan 180^\circ $
\vfill

\hfill 答.\underline{\hspace{50truemm}}

\item $\sin 90^\circ $
\vfill

\hfill 答.\underline{\hspace{50truemm}}

\item $\cos 120^\circ $
\vfill

\hfill 答.\underline{\hspace{50truemm}}

\end{enumerate}

\item 次の三角比を、(  )内の大きさの角の三角比で表せ。

\begin{enumerate}
\item $\sin 78^\circ$ \hfill ($45^\circ$以下) \hspace{30truemm}
\vfill

\hfill 答.\underline{\hspace{50truemm}}

\item $\cos 71^\circ$ \hfill ($45^\circ$以下) \hspace{30truemm}
\vfill

\hfill 答.\underline{\hspace{50truemm}}

\end{enumerate}

\newpage
\item $\triangle$ABCについて、次の問いに答えよ。

\begin{enumerate}
\item $\cos \theta =\nfrac{1}{2}のとき、\sin \theta 、\tan \theta の値を求めよ。(0^\circ \le \theta \le 90^\circ )$
\vfill

\hfill 答.$\sin \theta$=\underline{\hspace{25truemm}}\,,\,$\tan \theta$=\underline{\hspace{25truemm}}

\item $\cos \theta =\nfrac{1}{2}のとき、\sin \theta 、\tan \theta の値を求めよ。(0^\circ \le \theta \le 180^\circ )$
\vfill

\hfill 答.$\sin \theta$=\underline{\hspace{25truemm}}\,,\,$\tan \theta$=\underline{\hspace{25truemm}}


\end{enumerate}

\item 次のような三角形ABCで、(  )内の値を求めよ。

\begin{enumerate}
\item $a=5\sqrt{\,2\,} \,,\, b=5\sqrt{\,2\,} \,,\, A=30^\circ  \hfill (\,\sin B\,) \hspace{30truemm}$
\vfill

\hfill 答.\underline{\hspace{50truemm}}

\end{enumerate}

\item 次のような三角形ABCで、(  )内の値を求めよ。

\begin{enumerate}
\item $a=4 \,,\, b=\sqrt{\,3\,} \,,\, C=150^\circ  \hfill (\,c\,) \hspace{30truemm}$
\vfill

\hfill 答.\underline{\hspace{50truemm}}

\end{enumerate}

\item 次のような三角形ABCの面積を求めよ。

\begin{enumerate}
\item $a=1 \,,\, b=5 \,,\, C=30^\circ $
\vfill

\hfill 答.\underline{\hspace{50truemm}}

\end{enumerate}

\end{enumerate}

\newpage

\twocolumn[
{\Large \bf 三角比のまとめ4の解答}
\hfill
    年  組  番 氏名\underline{\hspace{50truemm}}
\vspace{1zh}]

\begin{enumerate}
\item 次の三角比の値を求めよ。

\begin{enumerate}
\item $\cos 90^\circ $
\vfill

\hfill 答.$0$

\item $\sin 120^\circ $
\vfill

\hfill 答.$\nfrac{\sqrt{\,3\,}}{2}$

\item $\cos 0^\circ $
\vfill

\hfill 答.$1$

\item $\tan 180^\circ $
\vfill

\hfill 答.$0$

\item $\sin 90^\circ $
\vfill

\hfill 答.$1$

\item $\cos 120^\circ $
\vfill

\hfill 答.$-\nfrac{1}{2}$

\end{enumerate}

\item 次の三角比を、(  )内の大きさの角の三角比で表せ。

\begin{enumerate}
\item $\sin 78^\circ$ \hfill ($45^\circ$以下) \hspace{30truemm}
\vfill

\hfill 答.$\cos 12^\circ$

\item $\cos 71^\circ$ \hfill ($45^\circ$以下) \hspace{30truemm}
\vfill

\hfill 答.$\sin 19^\circ$

\end{enumerate}

\newpage
\item $\triangle$ABCについて、次の問いに答えよ。

\begin{enumerate}
\item $\cos \theta =\nfrac{1}{2}のとき、\sin \theta 、\tan \theta の値を求めよ。(0^\circ \le \theta \le 90^\circ )$
\vfill

\begin{flushright}
答.$\sin \theta =\nfrac{\sqrt{\,3\,}}{2}\,,\,\tan \theta =\sqrt{\,3\,}$
\end{flushright}

\item $\cos \theta =\nfrac{1}{2}のとき、\sin \theta 、\tan \theta の値を求めよ。(0^\circ \le \theta \le 180^\circ )$
\vfill

\begin{flushright}
答.$\sin \theta =\nfrac{\sqrt{\,3\,}}{2}\,,\,\tan \theta =\sqrt{\,3\,}$
\end{flushright}


\end{enumerate}

\item 次のような三角形ABCで、(  )内の値を求めよ。

\begin{enumerate}
\item $a=5\sqrt{\,2\,} \,,\, b=5\sqrt{\,2\,} \,,\, A=30^\circ  \hfill (\,\sin B\,) \hspace{30truemm}$
\vfill

\hfill 答.$\nfrac{1}{2}$

\end{enumerate}

\item 次のような三角形ABCで、(  )内の値を求めよ。

\begin{enumerate}
\item $a=4 \,,\, b=\sqrt{\,3\,} \,,\, C=150^\circ  \hfill (\,c\,) \hspace{30truemm}$
\vfill

\hfill 答.$\sqrt{\,31\,}$

\end{enumerate}

\item 次のような三角形ABCの面積を求めよ。

\begin{enumerate}
\item $a=1 \,,\, b=5 \,,\, C=30^\circ $
\vfill

\hfill 答.$\nfrac{5}{4}$

\end{enumerate}

\end{enumerate}

\end{document}






